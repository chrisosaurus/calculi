% note 'oneside' here prevents blank pages
% by default book tries to make every chapter start on odd (left side)
% so it works better for double-sided books
% see http://stackoverflow.com/a/8371473/337237
\documentclass[11pt,oneside]{book}

\usepackage[margin=1in]{geometry}
\usepackage{amssymb}
\def\emptyset{$\varnothing$}
\def\natnumbers{$\mathbb{N}$}
\usepackage{bussproofs}
\usepackage{listings}

% had to add nounderscore when I moved files out into inputs
% see https://tex.stackexchange.com/a/166630
\usepackage[nounderscore]{syntax}

% custom macro to insert labels into grammar
\newcommand{\synlabel}[1]{\hfill \textit{#1}}

\begin{document}

\frontmatter

\mainmatter

\pagebreak

\begin{center}
    {
        \bf
        \huge
        Lambda System F
    }
\end{center}

\begin{center}
{
    \bf
    \huge
    Grammar
}
\end{center}

% taken from https://tex.stackexchange.com/a/24934
\setlength{\grammarindent}{10em} % increase separation between LHS/RHS

% found in http://texdoc.net/texmf-dist/doc/latex/mdwtools/syntax.pdf
\setlength{\grammarparsep}{5pt} % default is 8pt, decrease sep, between rules

\textbf{Expressions}

\begin{grammar}

    <e> ::=  x                            \synlabel{variable}
        \alt $\lambda$ x:<T>.<e>          \synlabel{abstraction}
        \alt $<e>_1$ $<e>_2$              \synlabel{application}
        \alt $\Lambda$ X.<e>              \synlabel{type abstraction}
        \alt e [T]                        \synlabel{type application}
        \alt unit                         \synlabel{unit literal}
        \alt true                         \synlabel{true literal}
        \alt false                        \synlabel{false literal}
        \alt if $<e>_1$ $<e>_2$ $<e>_3$   \synlabel{if-else}

\end{grammar}

\textbf{Values}

\begin{grammar}

    <v> ::=  $\lambda$ x.<e>            \synlabel{abstraction}
        \alt $\Lambda$ X.<e>            \synlabel{type abstraction}
        \alt unit                       \synlabel{unit literal}
        \alt true                       \synlabel{true literal}
        \alt false                      \synlabel{false literal}

\end{grammar}

\textbf{Types}

\begin{grammar}

    <T> ::= <T> $\rightarrow$ <T>     \synlabel{function type}
        \alt X                        \synlabel{type variable}
        \alt $\forall$X.<T>           \synlabel{universal type}
        \alt Unit                     \synlabel{unit type}
        \alt Bool                     \synlabel{bool type}

\end{grammar}

\textbf{Typing Contexts}

\begin{grammar}

    <$\Gamma$> ::=  $\varnothing$                          \synlabel{empty context}
               \alt \syntleft $\Gamma$\syntright, x:<T>    \synlabel{term variable binding}
               \alt \syntleft $\Gamma$\syntright, X        \synlabel{type variable binding}

\end{grammar}

\textbf{Evaluation Contexts}

\begin{grammar}

    <C[\textbullet]> ::=  \textbullet                          \synlabel{default}
                     \alt <C[\textbullet]> $<e>_2$             \synlabel{application left}
                     \alt $<v>_1$ <C[\textbullet]>             \synlabel{application right}
                     \alt if <C[\textbullet]> $<e>_2$ $<e>_3$  \synlabel{if-else condition}
                     \alt <C[\textbullet]> [$<T>$]             \synlabel{type application}

\end{grammar}

\break

\hfill
\begin{center}
{
    \bf
    \huge
    Operational semantics
}
\end{center}

\textbf{Syntactic domains}

\begin{tabular}{ l l }
    \textit{e} & Expression \\
    \textit{v} & Values \\
\end{tabular}

\hfill\break

\textbf{Operational semantics}

\begin{center}
\framebox{
                $
                e
                \quad
                \rightarrow
                \quad
                e'
                $
}
\end{center}

\begin{prooftree}
    \AxiomC{ $
                e
                \quad
                \rightarrow
                \quad
                e'
             $ }
    \RightLabel{\it context}
    \UnaryInfC{ $
                C[e]
                \quad
                \rightarrow
                \quad
                C[e']
                $ }
\end{prooftree}

\begin{prooftree}
    \AxiomC{ $
             $ }
    \RightLabel{\it application}
    \UnaryInfC{ $
                (\lambda x : T . e_1) v_2
                \quad
                \rightarrow
                \quad
                e_1 [v_2 / x]
                $ }
\end{prooftree}

\begin{prooftree}
    \AxiomC{ $
             $ }
    \RightLabel{\it type application}
    \UnaryInfC{ $
                (\Lambda X . e_1) [T]
                \quad
                \rightarrow
                \quad
                e_1 [T / X]
                $ }
\end{prooftree}

\begin{prooftree}
    \AxiomC{ $
                e_1
                \quad
                \rightarrow
                \quad
                true
             $ }
    \RightLabel{\it if-true}
    \UnaryInfC{ $
                if \ e_1 \ e_2 \ e_3
                \quad
                \rightarrow
                \quad
                e_2
                $ }
\end{prooftree}

\begin{prooftree}
    \AxiomC{ $
                e_1
                \quad
                \rightarrow
                \quad
                false
             $ }
    \RightLabel{\it if-false}
    \UnaryInfC{ $
                if \ e_1 \ e_2 \ e_3
                \quad
                \rightarrow
                \quad
                e_3
                $ }
\end{prooftree}

\hfill
\begin{center}
{
    \bf
    \huge
    Typing semantics
}
\end{center}

\hfill\break

\textbf{Typing semantics}

\begin{center}
\framebox{
                $
                \Gamma
                \quad
                \vdash
                \quad
                e : T
                $
}
\end{center}

\begin{prooftree}
    \AxiomC{ $
                x:T
                \in
                \Gamma
             $ }
    \RightLabel{\it T-variable}
    \UnaryInfC{ $
                \Gamma
                \vdash
                x : T
                $ }
\end{prooftree}

\begin{prooftree}
    \AxiomC{ $
                \Gamma, x:T_1
                \vdash
                e : T_2
             $ }
    \RightLabel{\it T-abstraction}
    \UnaryInfC{ $
                \Gamma
                \vdash
                (\lambda x:T_1 . \ e) : T_1 \rightarrow T_2
                $ }
\end{prooftree}

\begin{prooftree}
    \AxiomC{ $
                \Gamma \vdash e_1 : T_1 \rightarrow T_2
                \quad
                \Gamma \vdash e_2 : T_1
             $ }
    \RightLabel{\it T-application}
    \UnaryInfC{ $
                \Gamma
                \vdash
                e_1 \ e_2 : T_2
                $ }
\end{prooftree}

\begin{prooftree}
    \AxiomC{ $
                \Gamma, X
                \vdash
                e : T
             $ }
    \RightLabel{\it T-type-abstraction}
    \UnaryInfC{ $
                \Gamma
                \vdash
                (\Lambda X . \ e) : \forall X . T
                $ }
\end{prooftree}

\begin{prooftree}
    \AxiomC{ $
                \Gamma
                \vdash
                e : \forall X.T_1
             $ }
    \RightLabel{\it T-type-application}
    \UnaryInfC{ $
                \Gamma
                \vdash
                e [T_2] : T_1 [ T_2 / X]
                $ }
\end{prooftree}

\begin{prooftree}
    \AxiomC{ $
             $ }
    \RightLabel{\it T-unit}
    \UnaryInfC{ $
                unit : Unit
                $ }
\end{prooftree}

\begin{prooftree}
    \AxiomC{ $
             $ }
    \RightLabel{\it T-true}
    \UnaryInfC{ $
                true : Bool
                $ }
\end{prooftree}

\begin{prooftree}
    \AxiomC{ $
             $ }
    \RightLabel{\it T-false}
    \UnaryInfC{ $
                false : Bool
                $ }
\end{prooftree}

\begin{prooftree}
    \AxiomC{ $
                \Gamma \vdash e_1 : Bool
                \quad
                \Gamma \vdash e_2 : T
                \quad
                \Gamma \vdash e_3 : T
             $ }
    \RightLabel{\it T-if}
    \UnaryInfC{ $
                \Gamma \vdash
                if \ e_1 \ e_2 \ e_3
                :
                T
                $ }
\end{prooftree}

\end{document}

